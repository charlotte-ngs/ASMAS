\documentclass{scrartcl}

%\usepackage{fancyhdr}

\newcommand{\points}[1]
{\begin{flushright}\textbf{#1}\end{flushright}}

\usepackage{Sweave}
\begin{document}
\Sconcordance{concordance:asmas_w15_exam.tex:asmas_w15_exam.Rnw:%
1 7 1 1 0 1 1 1 33 97 1 1 10 21 0 1 2 32 1 2 2 71 1}



\thispagestyle{empty}

\titlehead
{
	ETH Z\"urich\\%
	D-USYS\\%
	Institut f\"ur Agrarwissenschaften
}

\title{\vspace{5ex} Pr\"ufung \\
       Angewandte Statistische Methoden in den Nutzierwissenschaften\\
       FS 2016 \vspace{3ex}}
\author{Peter von Rohr \vspace{3ex}}
\date{
  \begin{tabular}{lr}
  \textsc{Datum}  & \textsc{\emph{30. Mai 2016}} \\
  \textsc{Beginn} & \textsc{\emph{08:00 Uhr}}\\
  \textsc{Ende}  & \textsc{\emph{08:45 Uhr}}\vspace{3ex}
\end{tabular}}
\maketitle

% Table with Name
\begin{tabular}{p{3cm}p{6cm}}
Name:     &  \\
         &  \\
Legi-Nr:  & \\
\end{tabular}

% Table with Points

\vspace{3ex}
\begin{center}
\begin{tabular}{|p{3cm}|c|c|}
\hline
Aufgabe  &  Maximale Punktzahl  &  Erreichte Punktzahl\\
\hline
1        &  10  & \\
\hline
2        &  13  & \\
\hline
3        &  6  & \\
\hline
4        &  6  & \\
\hline
Total    &  35 & \\
\hline
\end{tabular}
\end{center}

\clearpage
\pagebreak

\section*{Aufgabe 1: Modellierung vor und nach Einf\"uhrung der Genomischen Selektion}
\begin{enumerate}
\item[a)] Wo liegen die Unterschiede im Bezug auf die Modellierung von Tierzuchtdaten vor und nach der Einf\"uhrung der genomischen Selektion (GS) im Bezug auf die folgendne Punkte?
\end{enumerate}
\points{6}

\vspace{5ex}
\begin{tabular}{|l|p{6cm}|p{6cm}|}
\hline
Punkt  &  vor GS  &  nach GS \\
\hline
Informationsquellen   &
                      &  \vspace{20ex}\\
\hline
statistisches Modell  &
                      &  \vspace{25ex}\\
\hline
genetisches Modell    &
                      &  \vspace{25ex}\\
\hline
\end{tabular}

\clearpage
\pagebreak

\begin{enumerate}
\item[b)] In der genomischen Selektion werden h\"aufig gesch\"atzte BLUP-Zuchtwerte aus einem Tiermodell als Beobachtungen verwendet.
\end{enumerate}

\begin{itemize}
\item Nennen sie je einen Vorteil und einen Nachteil der Verwendung von BLUP Zuchtwerten als Beobachtungen \vspace{30ex}
\item Welches Verfahren wird verwendet, um den Nachteil von der Verwendung von BLUP-Zuchtwerten als Beobachtungen, zu beheben und nach welchem Prinzip funktioniert dieses Verahren?



\end{itemize}
\points{4}


\clearpage
\pagebreak

\section*{Aufgabe 2: Lineare Regression}
Gegeben sind die folgenden Resultate einer linearen Regression
\begin{Schunk}
\begin{Soutput}
Call:
lm(formula = y ~ snp1 + snp2, data = dfSnpData)

Residuals:
     Min       1Q   Median       3Q      Max 
-11.6819  -2.9583   0.1485   2.7452   8.4649 

Coefficients:
            Estimate Std. Error t value Pr(>|t|)    
(Intercept)   0.9661     1.6819   0.574    0.570    
snp1         -2.3806     1.0970  -2.170    0.039 *  
snp2          6.5272     0.9994   6.531 5.28e-07 ***
---
Signif. codes:  0 ‘***’ 0.001 ‘**’ 0.01 ‘*’ 0.05 ‘.’ 0.1 ‘ ’ 1

Residual standard error: 4.999 on 27 degrees of freedom
Multiple R-squared:  0.6179,	Adjusted R-squared:  0.5896 
F-statistic: 21.83 on 2 and 27 DF,  p-value: 2.286e-06
\end{Soutput}
\end{Schunk}

\begin{enumerate}
\item[a)] Aus welchen Komponenten besteht das lineare Modell?
\end{enumerate}
\points{3}


\clearpage
\pagebreak

\begin{enumerate}
\item[b)] Wie sieht das Modell aus, welches zu den oben gezeigten Resultaten gef\"uhrt hat?
\end{enumerate}
\points{5}


\clearpage
\pagebreak

\begin{enumerate}
\item[c)] Berechnen Sie aus den oben gezeigten Resultaten das Vertrauensinterval f\"ur die erkl\"arende Variable \verb+snp1+. Wie gross ist die Irrtumswahrscheinlichkeit f\"ur dieses Vertrauensintervall?
\end{enumerate}
\points{3}


\clearpage
\pagebreak

\begin{enumerate}
\item[d)] Wie heisst der folgende Plot und wozu kann dieser Plot verwendet werden?
\end{enumerate}
\points{2}

\includegraphics{asmas_w15_exam-TukeyAnscombePlot}




\clearpage
\pagebreak

\noindent\textbf{Zusatz:} Durch welches Statement wird der oben gezeigte Plot in R erzeugt?

\points{2}



\clearpage
\pagebreak

\section*{Aufgabe 3: LASSO}

\begin{enumerate}
\item[a)] Was bedeutet die Abk\"urzung LASSO?
\end{enumerate}
\points{1}


\clearpage
\pagebreak

\begin{enumerate}
\item[b)] Sobald in einem linearen Modell die Anzahl Parameter gr\"osser ist als die Anzahl Beobachtungen k\"onnen wir Least Squares nicht verwenden. Was sind in einem solchen Fall Alternativen zu Least Squares?
\end{enumerate}
\points{3}


\clearpage
\pagebreak

\begin{enumerate}
\item[c)] Wie unterscheiden sich die Sch\"atzer durch Least Squares vom Sch\"atzer durch LASSO und wie wird die Selektion der Variablen erreicht?
\end{enumerate}
\points{2}



\clearpage
\pagebreak

\section*{Aufgabe 4: Bayes}

\begin{enumerate}
\item[a)] In welche Kategorien unterteilen Bayesianer die Komponenten eines Modells?
\end{enumerate}
\points{2}


\clearpage
\pagebreak

\begin{enumerate}
\item[b)] Worauf basieren Sch\"atzungen in der Bayes'schen Statistik, aus welchen Komponente besteht das gesuchte Objekt und wie wird dieses berechnet?
\end{enumerate}
\points{3}

\clearpage
\pagebreak

\begin{enumerate}
\item[c)] Angenommen, Sie haben vor der Sch\"atzung eines Parameters keine Information \"uber den Parameter. Wie lassen Sie diese Tatsache in einer Bayes'schen Analyse einfliessen?
\end{enumerate}
\points{1}


\end{document}
